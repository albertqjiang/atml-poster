%%%%%%%%%%%%%%%%%%%%%%%%%%%%%%%%%%%%%%%%%
% A beamer poster style for the University of Oxford. Atilim Gunes Baydin <gunes@robots.ox.ac.uk>, November 2016.
% Based on the I6pd2 style created by Thomas Deselaers an Philippe Dreuw.
%
% Dreuw & Deselaer's Poster
% LaTeX Template
% Version 1.0 (11/04/13)
%
% Created by:
% Philippe Dreuw and Thomas Deselaers
% http://www-i6.informatik.rwth-aachen.de/~dreuw/latexbeamerposter.php
%
% This template has been downloaded from:
% http://www.LaTeXTemplates.com
%
% License:
% CC BY-NC-SA 3.0 (http://creativecommons.org/licenses/by-nc-sa/3.0/)
%
%%%%%%%%%%%%%%%%%%%%%%%%%%%%%%%%%%%%%%%%%

%----------------------------------------------------------------------------------------
%   PACKAGES AND OTHER DOCUMENT CONFIGURATIONS
%----------------------------------------------------------------------------------------

\documentclass[final,hyperref={pdfpagelabels=false}]{beamer}

\usepackage[orientation=landscape,size=a0,scale=1.3]{beamerposter} % Use the beamerposter package for laying out the poster with a portrait orientation and an a0 paper size

\usetheme{Oxford}

\usepackage[utf8]{inputenc} % allow utf-8 input
\usepackage{blindtext}
\usepackage{amsmath,amsthm,amssymb,latexsym} % For including math equations, theorems, symbols, etc
\usepackage[document]{ragged2e}
\usepackage{times}\usefonttheme{professionalfonts}  % Uncomment to use Times as the main font
\usefonttheme[onlymath]{serif} % Uncomment to use a Serif font within math environments
%\boldmath % Use bold for everything within the math environment
\usepackage{booktabs} % Top and bottom rules for tables
\usepackage{microtype}
\usepackage{subcaption}
\usepackage{tcolorbox}

% custom coloured box
\definecolor{darkgreen}{rgb}{0,.7,0}
\definecolor{darkred}{rgb}{0.54,0,0}
\definecolor{camblue}{rgb}{0.639,0.757,0.678}

\usecaptiontemplate{\small\structure{\insertcaptionname~\insertcaptionnumber: }\insertcaption} % A fix for figure numbering

\newcommand{\shrink}{-15pt}

\def\imagetop#1{\vtop{\null\hbox{#1}}}

\let\oldbibliography\thebibliography
\renewcommand{\thebibliography}[1]{\oldbibliography{#1}
\setlength{\itemsep}{-10pt}}

%----------------------------------------------------------------------------------------
%   TITLE SECTION 
%----------------------------------------------------------------------------------------
\title{ATML Paper Reproduction Challenge} % Poster title
\author{Jad Ghalayini, Albert Qiaochu Jiang, Kamilė Stankevičiūtė}
\institute{Department of Computer Science, University of Oxford\\\vspace{4mm}
\texttt{\{jad.ghalayini,qiaochu.jiang,kamile.stankeviciute\}@cs.ox.ac.uk}}

%----------------------------------------------------------------------------------------
%   FOOTER TEXT
%----------------------------------------------------------------------------------------
\newcommand{\leftfoot}{} % Left footer text
\newcommand{\rightfoot}{} % Right footer text


%----------------------------------------------------------------------------------------

\begin{document}
\addtobeamertemplate{block end}{}{\vspace*{2ex}} % White space under blocks

\begin{frame}[t] % The whole poster is enclosed in one beamer frame

\begin{columns}[t] % The whole poster consists of three major columns, each of which can be subdivided further with another \begin{columns} block - the [t] argument aligns each column's content to the top

  \begin{column}{.02\textwidth}\end{column} % Empty spacer column

%%%%%%%%%%%%%%%%%%%%%%%%%%%%%%%%%%%%%%%%%%
%% Column 1
%%%%%%%%%%%%%%%%%%%%%%%%%%%%%%%%%%%%%%%%%%

  \begin{column}{.3\textwidth} % 1st column
    \vspace{\shrink}          
    \begin{block}{Selected Paper: Gated Graph Sequence Neural Networks~\cite{DBLP:journals/corr/LiTBZ15}}
    
      \begin{itemize}
          \item Introducing new architectures for graph representation learning.
          \item Benchmarking the newly introduced GNNs against baseline RNNs on bAbI tasks and graph algorithm learning tasks.
          \item Demonstrating the practical use of new architectures for program verification.
      \end{itemize}
      
      
      The introduced architectures are \vspace{0.2in}
      \begin{tcolorbox}[colframe=white, colback=camblue!20]{
      \textcolor{darkred}{Gated Graph Neural Networks~(GGNN)} for making single-step predictions.
      \vspace{0.2in}
      
      \textcolor{darkred}{Gated Graph Sequence Neural Networks~(GGSNN)} for making multiple-step predictions.}
        
        \end{tcolorbox}
    \end{block}
    
    
    
    \vspace{\shrink} 
    \begin{block}{Recurrent Graph Neural Networks}
    Use recurrence to model propagation
    
      \begin{figure}
        \centering
        \begin{subfigure}{.25\textwidth}
          \centering
          \includegraphics[height=3in]{imgs/example-graph.pdf}
          \caption{}
          \label{fig:sub1}
        \end{subfigure}%
        \hfill
        \begin{subfigure}{.43\textwidth}
          \centering
          \includegraphics[height=3in]{imgs/recurrent-matrix-sparsity-pattern2.pdf}
          \caption{}
          \label{fig:sub2}
        \end{subfigure}
        \begin{subfigure}{.25\textwidth}
          \centering
          \includegraphics[height=3in]{imgs/unrolled-graph3.pdf}
          \caption{}
          \label{fig:sub2}
        \end{subfigure}
        
        \caption{(a) An example of a directed graph. Colour denotes the type of the edge. (b) Adjacency matrix with edge typing. (c) Message passing by unrolling recurrence for one timestep. Dotted lines denote propagation in reverse edge directions. This figure is a re-ordered duplicate of Figure 1 from \cite{DBLP:journals/corr/LiTBZ15}.}
        \label{fig:test}
      \end{figure}
      
      
    \end{block}
    
     \vspace{\shrink} 
    \begin{block}{Gated Graph Neural Networks}
    We explain the mechanism implemented by a GGNN on a graph $\mathcal{G}=(\mathcal{V}, \mathcal{E})$, where $\mathcal{V}$ is the set of all the nodes, and $\mathcal{E}$ is the set of all the edges. For an arbitrary directed edge of type $T$, we also include a directed edge of type $T'$ in $\mathcal{E}$.
    \vspace{0.2in}
    
    \textcolor{darkred}{Initialisation}\quad The latent representations for graph nodes are initialised by padding the node annotations:
    \begin{equation}
        \forall v \in \mathcal{V}. \quad \mathbf{h}_v^{0} = [\mathbf{x}_v^T, \mathbf{0}]^T
    \end{equation}
    
    
    \textcolor{darkred}{Message Passing}\quad For an edge $u \to v$ of type $A$, let $\mathbf{h}_u^t$, $\mathbf{h}_v^t$ be the latent representations for $u$ and $v$ at the $t$-th step, and $\mathbf{A}$ be the edge type matrix associated with $A$. The message to pass from $u$ to $v$ is
    \begin{equation}
        \mathbf{m}_{uv}^t = \mathbf{A} \mathbf{h}_u^t.
    \end{equation}
    
    \textcolor{darkred}{Propagation}\quad To aggregate messages from its neighbours, each node updates its representation by 
    
    
    
    \end{block}
    
    
    
    
  \end{column} % End of the 1st column

%%%%%%%%%%%%%%%%%%%%%%%%%%%%%%%%%%%%%%%%%%
%% Column 2
%%%%%%%%%%%%%%%%%%%%%%%%%%%%%%%%%%%%%%%%%%

  \begin{column}{.02\textwidth}\end{column} % Empty spacer column

  \begin{column}{.3\textwidth} % 2nd column
    \vspace{\shrink}
    \begin{block}{Method}
      \textbf{Method details.} We formulate... $\mathcal S = \{s_1, s_2, s_3, s_4, s_5, s_6 \}$

      Image example:
      \begin{center}
        \includegraphics[width=0.9\columnwidth]{ox_brand_cmyk_rev}
      \end{center}
    \end{block}

  \end{column} % End of the 2nd column

%%%%%%%%%%%%%%%%%%%%%%%%%%%%%%%%%%%%%%%%%%
%% Column 3
%%%%%%%%%%%%%%%%%%%%%%%%%%%%%%%%%%%%%%%%%%

  \begin{column}{.02\textwidth}\end{column} % Empty spacer column

  \begin{column}{.3\textwidth} % 3rd column

    \begin{block}{Results}
      Results...
    \end{block}

    \begin{block}{References}
      \nocite{*} % Insert publications even if they are not cited in the poster
      \linespread{0.928}\selectfont
      \footnotesize{\bibliographystyle{unsrt}
      \bibliography{oxford_poster}}
    \end{block}

  \end{column} % End of the 3rd column

  \begin{column}{.02\textwidth}\end{column} % Empty spacer column

\end{columns} % End of all the columns in the poster

\end{frame} % End of the enclosing frame

\end{document}